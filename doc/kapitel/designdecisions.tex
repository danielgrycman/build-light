\section{Entwurfsentscheidungen}
%Inhalt
%------
%Dokumentieren Sie hier alle wesentlichen Entwurfsentscheidungen und deren Gründe!
%
%Motivation
%----------
%Es ist wünschenswert, alle wichtigen Entwurfsentscheidungen geschlossen nachlesen zu können. Wägen Sie ab, inwiefern Entwurfsentscheidungen hier zentral dokumentiert werden sollen oder wo eine lokale Beschreibung (z.B in der Whitebox-Sicht von Bausteinen) sinnvoller ist. Vermeiden Sie aber redundante Texte. Verweisen Sie evtl. auf Kap. 4 zurück, wo schon zentrale Architekturstrategien motiviert wurden.
%
%Form
%----
%informelle Liste, möglichst nach Wichtigkeit und Tragweite der Entscheidungen für den Leser aufgebaut.
%
%Alternativ auch ausführlicher in Form von einzelnen Unterkapiteln je Entscheidung. Die folgende Mindmap (Quelle: Kolumne „Architekturen dokumentieren“ von S. Zörner im Java Magazin 3/2009) soll Sie dabei unterstützen, wichtige Entscheidungen zu treffen und festzuhalten. Die Hauptäste stellen dabei die wesentlichen Schritte dar. Sie können auch als Überschriften innerhalb eines Unterkapitels dienen (siehe Beispiel unten).
%
%
%Die Fragen sind nicht sklavisch der Reihe nach zu beantworten. Sie sollen Sie lediglich leiten. In der Vorlage löschen Sie diese heraus, und lassen nur die Inhalte/Antworten stehen.

\subsection{Entwurfsentscheidung 1}

\subsubsection{Fragestellung}
%Was genau ist das Problem?
%Warum ist es für die Architektur relevant?
%Welche Auswirkung hat die Entscheidung?

\subsubsection{Rahmenbedingungen}
%Welche festen Randbedingungen haben Sie einzuhalten?
%Welche EInflussfaktoren sind zu beachten?

\subsubsection{Annahmen}
%Welche Annahmen haben Sie getroffen?
%Welche Annahmen können wie vorab überprüft werden?
%Mit welchen Risiken müssen Sie rechnen?

\subsubsection{Betrachtete Alternativen}
%Welche Lösungsoptionen ziehen Sie in die nähere Auswahl?
%Wie bewerten Sie jede einzelne?
%Welche Optionen schließen Sie bewusst aus?

\subsubsection{Entscheidung}
%Wer (wenn nicht Sie selbst) hat die Entscheidung getroffen?
%Wie ist sie begründet?
%Wann wurde entschieden?
